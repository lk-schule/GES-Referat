\documentclass{beamer}
\setbeamertemplate{navigation symbols}{}
\usepackage[T1]{fontenc}
\usepackage[utf8]{inputenc}

\usetheme{Montpellier}
\usepackage{ngerman}
\beamersetuncovermixins{\opaqueness<1>{25}}{\opaqueness<2->{15}}
\begin{document}
\title{Islamischer Staat}
\author{Luca Hartmann \and Luca Hülsmann \and Luca Kiebel}
\date{\today}

\begin{frame}
\titlepage
\end{frame}

\begin{frame}
\frametitle{Inhaltsverzeichnis}\tableofcontents
\end{frame}

\section{Geschichte}
\begin{frame}
\frametitle{Geschichte}
Die Geschichte des sog. Islamischen Staates von 2014 bis 2017
\end{frame}

\subsection{2014 - 2015}
\begin{frame}
\begin{itemize}
	\item Falludscha und Ramadi erstürmt
	\item Beratung um Invasion von USA und EU
	\item Blitzoffensive auf Mossul
	\item Raub von Geld und Ausrüstung i.W.v. 1,5 Milliarden USDollar
\end{itemize}
\end{frame}
\subsection{2015 - 2016}
\begin{frame}
\begin{itemize}
	\item Anti-IS-Koalition fliegt verstärkt Luftangriffe auf Mossul

\end{itemize}
\end{frame}
\subsubsection{2017}
\begin{frame}
\begin{itemize}
	\item{2017}
\end{itemize}
\end{frame}


\section{Aktuell kontrollierte Gebiete}
\begin{frame}
\frametitle{Aktuell kontrollierte Gebiete}
Laut Iraks Ministerpräsidenten Haider al-Abadi sei der IS aus militärischer Sicht ausgelöscht.
Am Freitag, 17.11. hat die irakische Armee die letzte Stadt vom IS zurückerobert.
\end{frame}


\section{Struktur}
\begin{frame}
\frametitle{Struktur}
\end{frame}

% Hartmann
\section{Wichtige Personen}
\begin{frame}
\frametitle{Wichtige Personen}
\end{frame}


\section{Einfluss der westlichen Welt}
\begin{frame}
\frametitle{Einfluss der westlichen Welt}

\end{frame}

\section*{Quellen}
\begin{frame}
\frametitle{Quellen}
\begin{itemize}
	\item https://de.reuters.com/article/iran-is-idDEKBN1DL0PJ, abgerufen am 24.11.
	\item https://de.wikipedia.org/wiki/Islamischer\_Staat\_(Organisation), abgerufen am 24.11.
\end{itemize}
\end{frame}

\end{document}
